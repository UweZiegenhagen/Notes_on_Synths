 \documentclass[12pt,ngerman]{beamer}
 
 \usepackage[T1]{fontenc}
 \usepackage{booktabs}
 \usepackage{babel}
 \usepackage{graphicx}
 \usepackage{csquotes}
 \usepackage{xcolor}

\author{Uwe Ziegenhagen}
\title{Synthesizer}
 
 \begin{document}
 
\begin{frame}
\frametitle{Inhalt}

\tableofcontents

\end{frame}
 
\section{Theoretische Grundlagen}

\subsection{Ton und Klang}

\begin{frame}
\frametitle{Ton}

\begin{itemize}
	\item gleichmäßig und einheitliche Schwingung der Luft, die vom (menschlichen) Gehör wahrgenommen werden kann
	\item anders als ein Impuls (Hammerschlag, Knall)
	\item anders als ein Geräusch (ungleichmäßige Schwingungen und Frequenzen)
\end{itemize}\vspace*{1em}

Einzelne Töne werden charakterisiert nach \vspace*{0.5em}

\begin{itemize}
	\item Tonhöhe (Frequenz, Schwingungen pro Sekunde, Note)
	\item Tondauer (Sekunden oder Notenwert)
	\item Laut-/Tonstärke als Höhe der Amplitude, per Schalldruck in dB oder Lautstärkeangabe
\end{itemize}

\end{frame}


\begin{frame}
\frametitle{Klang}

\begin{itemize}
\item in der physikalischen Akustik: Klang = Ton
\item in der Musiktheorie das simultane Auftreten mehrerer Töne
\item Gemisch aus:

\begin{itemize}
	\item Grundton (1. Partialton)
	\item Obertönen
	\item Rauschanteilen
	\end{itemize}

\item Grundton bestimmt die wahrgenommene Tonhöhe
\item Obertöne bestimmen die Klangfarbe
\item Obertöne sind üblicherweise die ganzzahligen Vielfache des Grundtons (Kammerton\footnote{Stimmton/Normalton} a\textsuperscript{1} = 440 Hz, a\textsuperscript{2} = 880 Hz, a\textsuperscript{3} = 1320 Hz))
\end{itemize}
\end{frame}


\subsection{Synthese}

\begin{frame}
\frametitle{}


\begin{itemize}
\item 
\item 
\item 
\item 
\item 
\item 
\end{itemize}
\end{frame}
 
\section{Synthesizer - Geschichtliches}

\begin{frame}
\frametitle{Theremin und Co}


\begin{itemize}
\item 
\item 
\item 
\item 
\item 
\item 
\end{itemize}
\end{frame} 
 
\section{Arten von Synthesizern}
 
\begin{frame}
\frametitle{Modular, Semi-Modular}

\begin{itemize}
\item 
\item 
\item 
\item 
\item 
\item 
\end{itemize}
\end{frame}

\section{Kompakte Synthesizer}
 
\begin{frame}
\frametitle{}


\begin{itemize}
\item 
\item 
\item 
\item 
\item 
\item 
\end{itemize}
\end{frame}
 
\section{Modulare Synthesizer (mit VCVRack2)} 
 
\begin{frame}
\frametitle{}


\begin{itemize}
\item 
\item 
\item 
\item 
\item 
\item 
\end{itemize}
\end{frame} 
 
 \end{document}